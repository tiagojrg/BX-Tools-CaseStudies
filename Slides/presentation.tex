%----------------------------------------------------------------------------------------
%	PACKAGES AND THEMES
%----------------------------------------------------------------------------------------

\documentclass{beamer}

\mode<presentation> {

% The Beamer class comes with a number of default slide themes
% which change the colors and layouts of slides. Below this is a list
% of all the themes, uncomment each in turn to see what they look like.

%\usetheme{default}
%\usetheme{AnnArbor}
%\usetheme{Antibes}
%\usetheme{Bergen}
%\usetheme{Berkeley}
%\usetheme{Berlin}
%\usetheme{Boadilla}
%\usetheme{CambridgeUS}
\usetheme{Copenhagen}
%\usetheme{Darmstadt}
%\usetheme{Dresden}
%\usetheme{Frankfurt}
%\usetheme{Goettingen}
%\usetheme{Hannover}
%\usetheme{Ilmenau}
%\usetheme{JuanLesPins}
%\usetheme{Luebeck}
%\usetheme{Madrid}
%\usetheme{Malmoe}
%\usetheme{Marburg}
%\usetheme{Montpellier}
%\usetheme{PaloAlto}
%\usetheme{Pittsburgh}
%\usetheme{Rochester}
%\usetheme{Singapore}
%\usetheme{Szeged}
%\usetheme{Warsaw}

% As well as themes, the Beamer class has a number of color themes
% for any slide theme. Uncomment each of these in turn to see how it
% changes the colors of your current slide theme.

%\usecolortheme{albatross}
%\usecolortheme{beaver}
%\usecolortheme{beetle}
%\usecolortheme{crane}
%\usecolortheme{dolphin}
%\usecolortheme{dove}
%\usecolortheme{fly}
\usecolortheme{lily}
%\usecolortheme{orchid}
%\usecolortheme{rose}
%\usecolortheme{seagull}
%\usecolortheme{seahorse}
%\usecolortheme{whale}
%\usecolortheme{wolverine}

%\setbeamertemplate{footline} % To remove the footer line in all slides uncomment this line
%\setbeamertemplate{footline}[page number] % To replace the footer line in all slides with a simple slide count uncomment this line

%\setbeamertemplate{navigation symbols}{} % To remove the navigation symbols from the bottom of all slides uncomment this line
}

\usepackage{graphicx} % Allows including images
\usepackage{booktabs} % Allows the use of \toprule, \midrule and \bottomrule in tables
\usepackage[utf8]{inputenc}
\usepackage[T1]{fontenc}
\usepackage{amsmath}
\usepackage{multicol}

\usepackage{color}

\usepackage{url}

\renewcommand{\raggedright}{\leftskip=0pt \rightskip=0pt plus 0cm}

%----------------------------------------------------------------------------------------
%	TITLE PAGE
%----------------------------------------------------------------------------------------

\title[Comparing BX Tools with Examples]{Comparing BX Tools with Examples\\ } % The short title appears at the bottom of every slide, the full title is only on the title page
 
%\subtitle{FATBIT Workshop, Braga}

\author{	
FATBIT Workshop, Braga
} % Your name


\institute[DI.UM] % Your institution as it will appear on the bottom of every slide, may be shorthand to save space
{
HASLab, University of Minho \\ % Your institution for the title page
}
\date{October 3, 2013} % Date, can be changed to a custom date

\begin{document}

\begin{frame}
\titlepage % Print the title page as the first slide
\end{frame}

%\begin{frame}
%\frametitle{Overview} % Table of contents slide, comment this block out to remove it
%\tableofcontents % Throughout your presentation, if you choose to use \section{} and \subsection{} commands, these will automatically be printed on this slide as an overview of your presentation
%\end{frame}






%----------------------------------------------------------------------------------------
%	PRESENTATION SLIDES
%----------------------------------------------------------------------------------------


% Motivation ------------------------------------
\section{Motivation}

\begin{frame}
\frametitle{Motivation}

\end{frame}



% Approach ------------------------------------
\section{Approach}
\begin{frame}
\frametitle{Approach}

\end{frame}


% Tools to assess ------------------------------------
\section{Tools to assess}
\begin{frame}
\frametitle{Tools to assess}

\end{frame}


% Case Studies ------------------------------------



% Bijection ------------------------------------

\section{Case Studies}

\subsection{Bijection}

\begin{frame}
\frametitle{\textbf{Bijection} - \textbf{Metamodel + Consistency Relation}}

\end{frame}


% eMoflon ------------------------------------
\begin{frame}
\frametitle{Bijection - \textbf{Metamodel + Consistency Relation} - \textbf{\textit{\textcolor{orange}{eMoflon}}}}

\end{frame}

\begin{frame}
\frametitle{Bijection - \textbf{Transformations} - \textbf{\textit{\textcolor{orange}{eMoflon}}}}

\end{frame}


% echo ------------------------------------
\begin{frame}
\frametitle{Bijection - \textbf{Metamodel + Consistency Relation} - \textbf{\textit{\textcolor{green}{echo}}}}

\end{frame}

\begin{frame}
\frametitle{Bijection - \textbf{Transformations} - \textbf{\textit{\textcolor{green}{echo}}}}

\end{frame}








% uniNDTotal ------------------------------------

\subsection{uniNDTotal}

\begin{frame}
\frametitle{\textbf{uniNDTotal} - \textbf{Metamodel + Consistency Relation}}

\end{frame}


% eMoflon ------------------------------------
\begin{frame}
\frametitle{uniNDTotal - \textbf{Metamodel + Consistency Relation} - \textbf{\textit{\textcolor{orange}{eMoflon}}}}

\end{frame}

\begin{frame}
\frametitle{uniNDTotal - \textbf{Transformations} - \textbf{\textit{\textcolor{orange}{eMoflon}}}}

\end{frame}


% echo ------------------------------------
\begin{frame}
\frametitle{uniNDTotal - \textbf{Metamodel + Consistency Relation} - \textbf{\textit{\textcolor{green}{echo}}}}

\end{frame}

\begin{frame}
\frametitle{uniNDTotal - \textbf{Transformations} - \textbf{\textit{\textcolor{green}{echo}}}}

\end{frame}






% References ------------------------------------------------
\section{References}
\begin{frame}

\frametitle{References}


\begin{scriptsize}

\begin{thebibliography}{99} % Beamer does not support BibTeX so references must be inserted manually as below

\bibitem{Perdita}Stevens, Perdita. "Observations relating to the equivalences induced on model sets by bidirectional transformations." Electronic Communications of the EASST 49 (2012).

\bibitem{echo} Macedo, Nuno, and Alcino Cunha. "Implementing QVT-R bidirectional model transformations using Alloy." Fundamental Approaches to Software Engineering. Springer Berlin Heidelberg, 2013. 297-311.

%------------------ tools

\bibitem{eMoflon} eMoflon - \url{http://www.moflon.org/}

\bibitem{echo} echo - \url{https://github.com/haslab/echo}

\bibitem{GRoundTram} GRoundTram - \url{http://www.biglab.org/}

\bibitem{ModelMorf} ModelMorf - \url{http://www.tcs-trddc.com/}

\bibitem{medini} medini QVT - \url{http://projects.ikv.de/qvt}

\bibitem{focal} focal - \url{https://alliance.seas.upenn.edu/~harmony/old/}

\end{thebibliography}
\end{scriptsize}


\end{frame}



% Front Frame ------------------------------------------------

%\begin{frame}
%\titlepage
%\end{frame}

% End -------------------------------------------------------
% -------------------------------------------------------
% -------------------------------------------------------




\end{document} 
